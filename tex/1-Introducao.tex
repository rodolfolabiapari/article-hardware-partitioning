% !TEX root = ../main.tex
% !TeX encoding = UTF-8
\section{Introduction} \label{chap:introducao}

    %\todo[inline]{1) Contextualização: Apresente uma visão da área identificando a importância do contexto q está trabalhando. Introduza os "termos" mais importantes.}
    
    %O projeto de Sistemas Embarcados (SE) está cada dia mais complexo \cite{Jozwiak2017}. 
    %
    %A demanda por curto tempo para disponibilidade de produtos ao mercado somado ao fato de exigirem propriedades como alto desempenho, baixo consumo de energia e alocação de recursos, representam um desafio para projetistas de sistemas \wearables.
    The demand for the short time for availability of products to the market summed up to the fact of requiring proprieties as high performance, low energy consumption and resources allocation, represent a challenge for wearable systems designers.
    
    % wearable
    %Sistemas \Wearables,\ uma subcategoria de Sistemas Embarcados (SE), possuem o propósito de integrar-se ao sistema corporal, expandindo suas capacidades, criando uma integração cada vez mais intensa entre tecnologia e ser humano.
    Wearable Systems, a subcategory of Embedded Systems (ES), have the propose of integrating to the body system, expanding its capabilities, making an integration each more intense between technology and human being.
    %
    %Esses sistemas possuem diversos componentes implementados em \hs\ e ainda é um desafio combinar alto desempenho com baixo consumo de energia maximizando o tempo de uso \cite{Wolf1994, Edwards1994}.
    These systems have several components implemented in hardware and software, and it is still challenging to combine high performance with low energy consumption, maximizing the use time \cite{Wolf1994, Edwards1994}.
    %
    %Uma das maneiras de lidar com tais problemas consiste na combinação das funções do processador com os recursos dos Arranjo de Portas Programáveis em Campo (FPGAs, do inglês \textit{Field-Programmable Gates Array}) formando um sistema computacional híbrido.
    One of the ways of dealing with such problems consists of the combination of the processor functions with the resources of Field-Programmable Gates Array (FPGA) forming a hybrid computational system.
    
    
    %particionamento
    %Uma decisão que pode ser tomada em nível de implementação nestes sistemas é chamado de Particionamento \HS\ (também abreviado como particionamento) e tem se mostrado promissor aumentando o desempenho destes sistemas \cite{Sass2010, BenHajHassine2017}.
    A decision that can be taken in implementation level in these systems it is called Hardware and Software Partitioning (also abbreviated as partitioning) and has shown promise increasing performance those systems \cite{Sass2010, BenHajHassine2017}.
    
    %\todo[inline]{3) Descreva o estado da arte atual, sempre referenciando trabalhos importantes.}
    
    %Alguns trabalhos mostram que, uma implementação customizada em \hardware\ pode prover maior eficiência energética e \speedup, comparado à implementações em \software\ \cite{Zhang2008, BenHajHassine2017, Wolf1994, Canis2011, Stone2010}.
    Some works show that a custom implementation in hardware can provide higher energy efficiency and speedup, comparing to implementations in software \cite{Zhang2008, BenHajHassine2017, Wolf1994, Canis2011, Stone2010}.
    %
    %Em \cite{Jozwiak2017} \todo{[VJP] essa referência está no contexto do seu trabalho? Ou ela somente fala de dipositivos móveis e vestíveis? Se for genérica, sugiro retirar essa frase.}  exibe-se vários estudos sobre dispositivos móveis e \wearables\ e \cite{Trindade2016} afirma que um significante esforço foi posto na área de particionamento de SE nos últimos dez anos.
    
    
    %\todo[inline]{2) Gap: Quais são as questões em aberto, restrições e limitações do estado atual dessa pesquisa.}
    
    %Entretanto, mesmo com vários estudos relacionados à desempenho com particionamento de SE em plataformas FPGA, não existem estudos que avaliam a melhoria de desempenho especificamente para \wearables\ em plataformas FPGA.
    However, even with various researches related to performance with ES partitioning in FPGA platforms, there are not researches that evaluate performance improvement especially for wearables in FPGA platforms.
    
    
    %\todo[inline]{4) Propósito+metodologia: Descreva o propósito do seu artigo utilizando para isso uma pitada da sua metodologia.}
    
    %Esta pesquisa consiste no particionamento de alguns algoritmos candidatos dentro do \wearable\ comparando o desempenho, alocação de recursos e gasto energético de ambas as implementações \hs.
    This research consists of the partitioning of some candidates algorithms inside of the wearable comparing the performance, resources allocated and energy spent on both hardware and software implementations.
    %
    % combinação de fpga com cpu
    %Ao utilizar o FPGA é possível implementar um sistema e acelerá-lo usando recursos de \hardware\ por meio do particionamento, o que melhora o desempenho e eficiência energética \cite{Cong2009, Lo2009, Zhang2008a}.
    Utilizing the FPGA it is possible to implement a system and speed it up using hardware resources by partitioning, that increase the performance and energy efficiency \cite{Cong2009, Lo2009, Zhang2008a}.
    
    
    %\subsection{Contribuição}
    %parece mais objetivo que contribuição
    %Esta pesquisa consiste numa busca sobre o aprimoramento de desempenho de dispositivos computacionais \wearables\ em \hardwares\ reconfiguráveis, utilizando particionamento \hs\ como meio.
    %Visa gastos relativos ao uso de recursos em \hardware\ e gasto energético.
    %A principal contribuição deste trabalho é exibir que particionamento \hs\ é uma excelente técnica para a melhoria de desempenho de sistemas \wearables,\ como será exibido.
    The main contribution of this work it is shown that hardware and software partitioning it is an excellent technique for the performance improvement of wearable systems, as will be explained.
    
    %Adicionalmente, algumas contribuições específicas são listadas a seguir:
    Additionally, some specifics contributions are listed below:
    
    \begin{enumerate}
        \item 
        %Apresentação da modelagem do problema de particionamento \hs\ aplicando tal técnica nessa classe de sistemas embarcados, buscando pelo aprimoramento de desempenho;
        
        %Apresentação de uma modelagem do problema de particionamento aplicado à \wearables,\ buscando maior desempenho;
        Presentation of modeling of partitioning problem applied to wearables, targeting higher performance;
        
        \item 
        % wearables e particionamento
        %Introdução de sistemas computacionais \wearables\ na qual possuem restrições de consumo energético e recursos, utilizando uma plataforma FPGA como meio para análise de recursos alocados; 
        %Utilização de plataforma FPGA em sistemas \wearables\ com restrições energéticas e de recursos;
        Utilization of FPGA platform in wearable systems with resources and energy restrictions;
        
        
        \item %Obtenção de pelo menos $9,6\%$ a mais de desempenho em três de quatros algoritmos avaliados, alocando $5,5\%$ de recursos de \hardware\ reconfigurável e aumento de $ 5,4\% $ de gasto energético;
        %Análise de desempenho de quatro algoritmos utilizando particionamento em \hardware\ considerando alocação de recursos e consumo energético;
        Performance analysis of four algorithms using partitioning in hardware considering resources allocated and energy spent;
        
        \item 
        %Análise de como as interfaces de comunicação entre \hs\ e otimizações influenciam no desempenho dos \wearables.
        Analyse of how the communication interfaces between hardware and software and optimizations influence in wearable performance.
        
        %\item Os resultados mostram que com o uso da técnica de particionamento \hs\ em pedaços de código do \wearable,\ aumentou-se o desempenho do sistema pelo menos 2,2\%, chegando até em 41,6\% a mais em desempenho.
    \end{enumerate}

    %Avaliou-se algoritmos do sistema \wearable\ analisando o desempenho e alocação de cada um, sendo eles o Estatístico \Ss$_{Es}$, Lagrange \Ss$_{La}$, Números Primos \Ss$_{NP}$ e Risco \Ss$_{Ri}$ e para cada um obteve-se uma melhora de desempenho em relação à sua versão em \software\ de 2,2\%, $41,6\%$, $9,6\%$ e $17,8\%$ respectivamente.
    We evaluated algorithms of wearable system analyzing the performance and allocation of each one, begin them the Statistic \Ss$_{Es}$, Lagrange \Ss$_{La}$, Prime Numbers \Ss$_{NP}$ and Risk \Ss$_{Ri}$ and for each one was obtained a performance enhance related to its version in software of 2,2\%, 41,6\%, 9,6\% and 17,8\% respectively.
    
%    As próximas seções foram divididas da seguinte forma: 
%    Seção \ref{chap:revisao_bibliografica} apresenta a informações relevantes para o compreendimento e os trabalhos relacionados. 
%    Seção \ref{chap:design} exibe a metodologia utilizada.
%    Seção \ref{chap:prototipo} descreve o protótipo e o procedimento de testes.
%    Seção \ref{chap:results} exibe e analisa os resultados e a Seção \ref{chap:conclu} conclui e apresenta os trabalhos futuros.
%    
    The following sections were divided in this way:
    Section \ref{chap:revisao_bibliografica} presents information relevant to the understanding and related work.
    Section \ref{chap:design} shows the methodology used.
    Section \ref{chap:prototipo} describes the prototype and test procedure.
    Section \ref{chap:results} displays and analyzes the results and Section \ref{chap:conclu} concludes and presents future work.
    