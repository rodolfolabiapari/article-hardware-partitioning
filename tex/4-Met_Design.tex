%!TEX root = ../main.tex
% !TeX encoding = UTF-8
\section{Principles of Partitioning of General Systems} \label{chap:design}
    
    %A seguir serão descritos tópicos utilizados para o entendimento do particionamento, seguindo conceitos estabelecidos por vários trabalhos como \cite{Arato2003, Arato2005, Mann2007, BenHajHassine2017, Sass2010}.
    %Esses tópicos são necessários para o entendimento básico dos arcabouços metodológicos de \codesign\ e o seu particionamento para sistemas, em especial para \wearables, além de definições formais sobre.
    Next will be described topics used to the knowledge of partitioning, following concepts established by various works as \cite{Arato2003, Arato2005, Mann2007, BenHajHassine2017, Sass2010}.
    These topics are necessary for the fundamental understanding of the codesign methodological frameworks and its partitioning for systems, in particular for wearables, besides formal definitions.
    
    %É possível descrever sistemas livre de especificações formais por meio de descrições de protótipos simples, conhecidos como \Design\ de Referência de \Software\ (DRS)\cite{Sass2010}.
    %Dessa forma, o algoritmo a ser analisado também pode ser representado como grafo de rotinas, chamado de Grafo de Controle de Fluxo (GCF) \cite{Mann2007}.
    %Ele é definido por $C = (B, F) \label{eq:subrotinaa}$
    %onde $B$ são vértices que representam as tarefas 
    %%http://www.dcc.ufrj.br/~gabriel/microarq/Escalonamento.pdf
    %e $ F $ são arestas que indicam todas as possibilidades de caminhos entre elas.
    It is possible to describe systems free of formal specifications by simple prototype descriptions, known as software reference design (SRD) \cite{Sass2010}.
    In this way, the algorithm to be analyzed also can be represented by a routing graph, called Control Flow Graph (CFG) \cite{Mann2007}.
    It is defined by $C = (B, F) \label{eq:subrotina}$ where $B$ are vertices that represent the tasks and $ F $ are edges that represent all the possibility of the path between them.
    
    
    %A decomposição de um DRS pode gerar dois componentes: uma porção a ser realizada em \hardware\ e outra executada em \software.
    %Essa decisão de divisão é chamada de Problema de Particionamento.
    The decomposition of a SRD can generate two components: a portion to be executed in hardware and other in software.
    This division decision it is called of Problem of Partitioning Hardware and Software.
    
    \subsection{Formal Definition of the Hardware and Software Partitioning Problem} \label{sec:gc}
        %Um Grafo de Chamada (GC) consiste num conjunto de GCFs, um por sub-rotinas, ou seja, $\mathcal{C} = {C_0, C_1, \dots C_{n-1}}$
        A Call Graph (CG) consists in a conjunct of CFGs, one by subroutines, that is, $\mathcal{C} = {C_0, C_1, \dots C_{n-1}}$
        %onde $ C_i = (B_i, F_i) $ representa o GCF de uma sub-rotina $ i $.
        where $ C_i = (B_i, F_i) $ represents the CFG of one subroutine $ i $.
        %Sendo assim, o grafo estático de chamada da aplicação completa é escrito por $\mathcal{A} = (\mathcal{C}, \mathcal{L}) \label{eq:aa}$
        Therefore, the static call graph of all application is describes by $\mathcal{A} = (\mathcal{C}, \mathcal{L}) \label{eq:a}$
        %onde \A\ representa uma aplicação específica e $ \mathcal{L} \subseteq \mathcal{C} \times \mathcal{C} $ é um subconjunto do plano cartesiano\footnote{Duas sub-rotinas são relacionadas se podem ser determinadas que, no tempo de compilação, a sub-rotina $ i $ tem potencial de invocar a sub-rotina $ j $, ou seja, $ (C_i, C_j) \in \mathcal{L} $.} dos GCFs.
        where \A\ represents an specific application and $ \mathcal{L} \subseteq \mathcal{C} \times \mathcal{C} $ is a subset of cartesian plan\footnote{Two subroutines are related if it can be determined that, at compile time, the subroutine $ i $ has the potential to invoke the subroutine $ j $, that is, $ (C_i, C_j) \in \mathcal{L} $.}
        of CFGs.
        
        
        %Com esses conceitos definidos, é possível definir-se também uma partição como $ \mathcal{S} = \{S_0, S_1, \dots S_i\} $ sendo $ i $ quantidade de partições, e $U$ como o conjunto de todas as sub-rotinas.
        %Assim, uma partição \Ss\ de um conjunto universal $ U $ é definida como um conjunto de subconjuntos de $ U $, podendo ser definida como
        With these concepts defined, it is possible to define also a partition as $ \mathcal{S} = \{S_0, S_1, \dots S_i\} $ being $ i $ the quantity of partitions, and $ U $ as the set of all the subroutines.
        Thus, a partition \Ss\ of a universal set a $ U $ is defined as a set of subsets of $ U $, can be described as
        %
        \begin{eqnarray}
        \bigcup_{S \in \mathcal{S}} S &=& U           \label{eq:part_form_1} \\
        \forall S, S' \in \mathcal{S} | S \cap S' &=& \emptyset  \label{eq:part_form_2}\\
        \forall S \in \mathcal{S}\ |\ S &\neq& \emptyset  \label{eq:part_form_3}
        \end{eqnarray}
        %
        %onde a Equação~\ref{eq:part_form_1} diz que cada elemento de $ U $ é um membro de pelo menos um subconjunto $ S \in \mathcal{S} $, e as Equações~\ref{eq:part_form_2} e \ref{eq:part_form_3} dizem que os subconjuntos $ S \in \mathcal{S} $ são emparelhados, disjuntos e não vazios.
        %Em outras palavras, cada elemento do nosso universo $ U $ termina exatamente em um dos subconjuntos de $\mathcal{S}$ e nenhum dos subconjuntos são vazios.
        where the Equation~\ref{eq:part_form_1} says that each element of $ U $ is a member at least one subset $ S \in \mathcal{S} $, and the Equations~\ref{eq:part_form_2} and \ref{eq:part_form_3} say that the subsets $ S \in \mathcal{S} $ are paired, disjoint and non-empty.
        In others words, each element of our universe $ U $ ends precisely in one of the subsets \Ss\ and none of the subsets are empty.
        
        %Com isso, é possível aplicar o formalismo à $ \mathcal{A} $, se assumirmos que nosso universo é o conjunto de todas as tarefas $B$ de um dispositivo \wearable\ e, assim, $U$ são as partições de sub-rotinas
        With that, it is possible to apply the formalism to \A,\ if we assume that our universe is the set of all the tasks $ B $ of a wearable device and so, $ U $ are the partitions of subroutines.
        %
        \begin{equation}
        U = \bigcup_{C \in \mathcal{C}} B(C) \label{eq:bigcup}
        \end{equation}
        %
        %sendo esta a partição natural da aplicação, onde
        being this the natural partition of the application, where
        %
        \begin{equation} \small
        \mathcal{S}  = \left \{
        \underbrace{\left \{ b_0, b_1, \dots b_{i-1} \right \}}_{\textnormal{subroutine }C_0},
        \underbrace{\left \{ b_i, b_{i+1}, \dots \right \}}_{\textnormal{subroutine }C_1},\dots
        \underbrace{\left \{ b_j, b_{j+1}, \dots \right \}}_{\textnormal{subroutine }C_{n-1}}
        \right \}
        \end{equation}
        
        
        \makeatletter
        \def\@eqnnum{{\normalsize \normalcolor (\theequation)}}
        \makeatother
        
        %
        %O propósito é a reorganização de $U$ em um particionamento com dois subconjuntos  $ \mathcal{S} = \{S_0, S_1\} $, sendo eles \hs.
        %Dessa forma, teremos uma nova partição \Ss$'$ gerando um novo resultado $ \mathcal{A}’ = (\mathcal{C}’, \mathcal{L}’) $, inferido a partir da reorganização da partição $ \mathcal{S} $.
        The propose is the reorganization of $ U $ in a partitioning with two subsets $ \mathcal{S} = \{S_0, S_1\} $, being them hardware and software.
        Therefore, we will have a new partition \Ss$ ' $ generating a new result $ \mathcal{A}’ = (\mathcal{C}’, \mathcal{L}’) $, inferred by partition reorganization \Ss.
        
        %O segundo passo é mapear cada subconjunto de $ \mathcal{S}' $ para ambos \hs,\ como é exibido na Equação \ref{eq:part_final}.
        The second step is to map each subset of \Ss$ ' $ for both hardware and software, as is shown in Equation \ref{eq:part_final}.
        
        { \small
        \begin{equation}
        \mathcal{S}'\!=\!\left \{
        \underbrace{
        \underbrace{
        \left \{ b_0, b_1, ... b_{i-1} \right \}
        }_{\textnormal{subroutine }C_0},
        ...
        \underbrace{
        \left \{ b_j, b_{j+1}, ... \right \}
        }_{\textnormal{subroutine }C_{n-i}}
        ...
        }_{\textnormal{software}};
        \ 
        \underbrace{
        \underbrace{
        \left \{ b_i, b_{i+1}, ... \right \}
        }_{\textnormal{subroutine }C_{n-1}}
        ...
        }_{\textnormal{hardware}}
        \right \} \label{eq:part_final}
        \end{equation}
        }
        
    
    
    \subsection{Performance as Guide to the Hardware and Software Partitioning} \label{sec:ganho_performance}
        \subsubsection{Performance Gain}
        
            %Para aplicações em geral pode estar no acúmulo de pequenos ganhos.
            %Entretanto, diferente de algoritmos em \software\ no qual tem-se análise de ordem de complexidade, em \hardware\ não possui-se um guia geral para comparação.
            For applications, in general, the gain can be in the accumulation of small profits.
            However, different of software level algorithms which have time complexity analysis, in hardware level have not a general guide for comparing.
            
            
            % y é speedup
            %O ganho, ao comparar uma solução \hs\ contra uma solução puramente \software,\ é tipicamente mensurado por meio do \speedup, e pode ser obtido utilizando a Equação~\ref{eq:speedup1}.
            %Utiliza-se $ \gamma $ para sua representação, permitindo comparar recursos diferentes para determinar melhores particionamentos.
            The gain, when comparing a hardware and software solution against an only software solution is typically measured by speedup, and it can be obtained using the Equation~\ref{eq:speedup1}.
            It uses $ \gamma $ for its representation, allowing to compare different resources to determine betters partitionings.
            %
            \begin{equation}
            \gamma =
            \frac{
            \textnormal{\textit{hardware speed}}
            }{
            \textnormal{\textit{software speed}}
            }
            =
            \frac{
            \frac{
            1
            } {
            \textnormal{\textit{hardware time}}
            }
            } {
            \frac{
            1
            }{
            \textnormal{\textit{software time}}
            }
            }
            =
            \frac{
            \textnormal{\textit{software time}}
            } {
            \textnormal{\textit{hardware time}}
            } \label{eq:speedup1}
            \end{equation}
            %
            %Mais especificamente, interessa-se no ganho de desempenho individual de cada recurso que pode ser definido como $ \gamma(i), i \in \mathcal{C}\ |\ \gamma(i) =\ ^s\!/_h $.
            %Para que um algoritmo em \hardware\ seja mais performático, deseja-se que $\gamma > 1$.
            Specifically, it interest in the individual performance gain of each resource that can be defined as $ \gamma(i), i \in \mathcal{C}\ |\ \gamma(i) =\ ^s\!/_h $.
            For an algorithm in hardware be more performative, it is desired that $ \gamma > 1 $.
            
        
        \subsubsection{Finite Resources in Electronic Components} \label{sec:recursos}
        
            %Um FPGA terá um valor escalar total $ r_{FPGA} $, que representa o total de números de \luts\ disponíveis para sintetização.
            %Então $ r(i) $ pode ser usado para representar a quantidade de recursos requerida por cada algoritmo $ i $.
            %Fazendo uma simples relação, tem-se que $ \sum_{i \in \mathbb{D}} r(i) < r_{FPGA} $ restringe quão largo $ \mathbb{D} $ pode crescer, onde $ \mathbb{D} $ é o conjunto de algoritmos candidatos.
            A FPGA will have a scalar value total $ r_{FPGA} $, that represents the total available of Lookup Tables for synthetization.
            So, $ r(i) $ can be used to represent the number of resources required for each algorithm $ i $.
            Making a simple relation, we has that $ \sum_{i \in \mathbb{D}} r(i) < r_{FPGA} $ restricts how large $ \mathbb{D} $ can grow, where $ \mathbb{D} $ is the set of candidates algorithms.
            
            %Uma típica plataforma FPGA moderna tem múltiplos tipos de recursos além de \luts,\ como memória, blocos DSP, etc., sendo a quantidade de recursos de cada tecnologia representados matematicamente por  $r_{Logic\ Cells}$, $r_{Memory}$, $r_{DSP}$, $r_{n-1} $, respectivamente.
            %Dessa forma, podem ser representados por um vetor de recursos
            A typical modern FPGA platform has multiples types of resources besides Lookup Tables, as memory, DSP blocks, etc., being the number of resources of each technology represent mathematically by $r_{Logic\ Cells}$, $r_{Memory}$, $r_{DSP}$, $r_{n-1} $, respectively.
            So, they can be represented by a resource vector
            %
            %$ \vec{r}_{FPGA} = (r_{Logic\ Cells}, r_{Memory}, r_{DSP}, \dots r_{n-1} ) $
            %
            
                     \begin{equation} \footnotesize 
                        \vec{r}_{FPGA} =
                        \begin{pmatrix}
                        r_{Logic\ Cells} \\
                        r_{Memory}\\
                        r_{DSP}\\
                        \vdots \\
                        r_{n-1}
                        \end{pmatrix}
                     \end{equation}
            
            %e com isso,
            and with that,
            %
            %\begin{equation}
            $\sum_{i \in \mathbb{D}} \vec{r}(i) < \vec{r}_{FPGA}$.
            %\end{equation}
            %
    
    
    \subsection{Wearable's Evaluation}
        %\todo[inline]{ explicando como a teoria da secao III se aplica na pratica ao seu experimento.} 
        
        %Enquanto o desempenho do \wearable\ é avaliada pelo tempo de execução do sistema segundo a Equação~\ref{eq:speedup1}, a alocação de recursos deve ser analisada, 
        %tanto em cada um dos \hardwares\ dos algoritmos gerados por HLS, quanto para o sistema como um todo, respeitando o limite de recursos $ \vec{r}_{FPGA}$.
        While system runtime evaluates the wearable's performance according to Equation~\ref{eq:speedup1}, the resources allocation must be analyzed, both in each of the hardware of the algorithms generated by HLS and for the system as a whole, respecting the resource limit $ \vec{r}_{FPGA}$.
        
        %Dessa forma, será utilizado \A$_{i}$\ para descrever separadamente as implementações dos $i$ códigos candidatos e \Ss$_{j}$\ referencia-se às implementações dos $j$ sistemas sintetizados.
        %
        %Com a diferenciação entre \A$_{i}$\ e \Ss$_{j}$\ (parte e todo do sistema)\ é possível analisar tanto as implementações dos módulos gerados por HLS separadamente, quanto os sistemas finais na alocação de recursos.
        So, it will be used \A$_{i}$\ to describe separately the implementations of the candidate codes $ i $ and \Ss$_{j}$\ to reference the implementations of the synthesized systems $ j $.
        With the differentiation between \A$_{i}$\ and \Ss$_{j}$\ (part and all of the system), \ is possible to analyze both the implementations of the HLS modules separately and the final systems in resource allocation.
        
        