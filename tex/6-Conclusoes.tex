%!TEX root = ../main.tex
% !TeX encoding = UTF-8
\section{Conclusions} \label{chap:conclu}
    %projeto de sistemas
    %A demanda por curto tempo para disponibilidade ao mercado, somado ao fato dos produtos exigirem propriedades de corretude, rapidez, confiabilidade e preço acessível representam um desafio para projetistas de sistemas embarcados em geral.
    %utiliza o particionamento para o problema de desempenho
    %Com o desenvolvimento de sistemas embarcados cada vez mais complexos, o particionamento \hs\ tornou-se um problema de otimização em \codesign\ de sistemas.
    %wearable
    %Como dispositivos \wearables\ também demandam um alto desempenho e/ou baixo consumo de energia sem apresentar desequilíbrio em confiabilidade e segurança entre outros, aplicou-se o particionamento sobre essa classe de sistemas, com foco em tais melhorias.
    The short time demand for market disponibility summed up to the fact of products requires correctness, speed, reliability and affordable price proprieties represent a challenge for ES designers in general.
    With the ES development is increasingly complex, the hardware and software partitioning became an optimization problem in systems codesign.
    As wearable devices also require high performance and low energy cost without the tradeoff, was applied the partitioning in this system class targeting those improvements.
    
    %proposta
    %A proposta da pesquisa constituiu-se na busca pelo aprimoramento de desempenho de dispositivos computacionais \wearables,\ utilizando o particionamento como meio, visando gasto energético e de recursos limitados de plataforma FPGA.
    The propose of this research is to reach the enhancement of performance of wearable devices, using the partitioning, aiming energy spent and limited resources of FPGA platform.
    
    
    %comentando os testes
    %Para a avaliação, realizou-se particionamento de quatro algoritmos candidatos (Estatístico \A$_{St}$, Lagrange \A$_{La}$, Números Primos  \A$_{NP}$ e Processamento de Risco \A$_{Ri}$) de um projeto de capacete de segurança para ciclistas, variando cada teste em quantidade de sensores e também o \buffer\ de operação.
    For the evaluation, we partitioned four candidate algorithms (Statistical \A$_{St}$, Lagrange \A$_{La}$, Prime Numbers  \A$_{NP}$ and Risk Processing \A$_{Ri}$) of a helmet design for safety for cyclists, varying each test in sensors numbers and buffer size.
    
    % comentando os resultados
    %Três dos quatro sistemas (Lagrange \Ss$_{La}$, Números Primos \Ss$_{NP}$ e Risco \Ss$_{Ri}$) obtiveram sucesso na busca por desempenho apenas pelo processo de particionamento, aumentando no mínimo $9,6\%$ seus desempenhos, utilizando o valor máximo de $ 5,5\% $ de recursos e $ 5,4\% $ de energia do \hardware\ reconfigurável.
    %
    %O sistema Estatístico \Ss$_{St}$ em $45,5\%$ dos testes obteve maior desempenho \software\ comparado com \hardware.
    %Esse resultado já era esperado, já que seu código exige bastante comunicação entre \hs, afetando seu desempenho.
    Three of four systems (Lagrange \A$_{La}$, Prime Numbers  \A$_{NP}$ and Risk Processing \A$_{Ri}$) had success in reach for performance only by partitioning process, increasing at least 9,6\% of performance, using the maximum value of 5,5\% of resources and 5,4\% of the energy of reconfigurable hardware.
    The Statistical \Ss$_{St}$ in 45,5\% of tests had bigger performance in software than hardware.
    This result was already expected since its code requires more hardware and software communication, affecting its performance.

    %trabalhos futuros
    %har e softprocessadores
    %Para futuros trabalhos seria possível realizar a comparação do sistema \wearable\ particionado variando, também a arquitetura de sintetização, ou seja, testes de desempenho entre \textit{soft} e \textit{hard-}processadores, ambos utilizando FPGA.
    For future works we would compare the partitioned wearable system also varying its architecture, that is, comparing the wearable performance between soft and hard-processor, both in FPGA platform.
    
    % fpga e prototipações
    %Comparações entre o uso de plataformas FPGA e plataformas de prototipações também seriam viáveis, ainda com foco em busca em otimizações em desempenho e eficiência energética sobre \wearables.
    
    %mais alguma sugestão de trabalhos?
    %E também a adição do parâmetro otimização em \hardware,\ verificando o percentual de ganho ao utilizar as técnicas de otimizações em nível de \hardware\ existentes para HLS.
    
    
    % use section* for acknowledgment
    \section*{Acknowledgment}
        %Agradecemos à Universidade Federal de Ouro Preto, ao CNPq, CAPES e à FAPEMIG pelo subsídio dessa pesquisa.    
        We thank to the Federal University of Ouro Preto, to CNPq, CAPES ant the FAPEMIG by the subsidy of this research.

   % conference papers do not normally have an appendix
   %\section*{Apêndice}    
   %    \hphantom{a}
