%!TEX root = ../main.tex
% !TeX encoding = UTF-8
\subsection{Related Works}  \label{chap:relacionados}
    
    %Em \cite{Mei2000} além do particionamento, os autores apresentam uma abordagem de escalonamento para SE dinamicamente reconfigurável (DRESs, do inglês \textit{dynamically reconfigurable embedded systems}). 
    %Neste trabalho os autores fornecem análises dos tempos de configuração e reconfiguração parcial do FPGA e mostram que o algoritmo proposto resolve o problema de particionamento e escalonamento dos DRES.
    In \cite{Mei2000} besides of the partitioning, the authors present a scheduling approach for the dynamically reconfigurable ES (DRES).
    In this work, they provide analysis of configuration and partial reconfiguration time of FPGA and show that the algorithm proposed to solve the partitioning and scheduling problem for DRESs.
    
    %Em \cite{Arato2003} os autores descrevem algumas versões do problema de particionamento para sistemas de tempo real e custo restringido, provando que são problemas $ \mathcal{NP} $-difícil.
    %Adicionalmente, apresentam uma abordagem com programação linear inteira, resolvendo o problema de forma otimizada, e uma abordagem utilizando algoritmo genético, na qual encontram-se soluções próximas ao ótimo global.
    In \cite{Arato2003} the authors describe some versions of the partitioning problem for real-time systems and restricted cost, proving that they are $ \mathcal{NP} $-hard problems.
    Additionally, present an approach with integer linear programming, solving the problem of optimal way, and an approach using the genetic algorithm, which found solutions near to the global optimum.
    
    %O trabalho em \cite{Mann2007} descreve uma primeira tentativa para um algoritmo exato para o problema de particionamento.
    %Utiliza um esquema no qual implementa-se a estratégia \textit{branch-and-bound} como um \textit{framework}.
    %Eles demonstram que problemas de particionamento altamente complexos podem ser resolvidos em tempo razoável.
    The work in \cite{Mann2007} describes the first try for an exact algorithm for the partitioning problem.
    It uses the scheme which implements the branch-and-bound strategy as a framework.
    They demonstrate that partitioning problems highly complex can be resolved in a reasonable time.
    
    
    %Pesquisas mais recentes, como a de \cite{BenHajHassine2017} procuram aplicar otimizações sobre o tempo de execução e gasto energético para processadores baseados em produtos embarcados, por meio de particionamento.
    %Comparado com outras heurísticas, o algoritmo mostra-se ser mais adequado para aplicações que necessitam do equilíbrio no \textit{tradeoff}.
    Researches more recent as the \cite{BenHajHassine2017} target to apply optimizations on execution time and energy spent on processors based in embedded products, by partitioning.
    Comparing with others heuristics, the algorithm show to be more suitable for applications that need of equilibrium in the tradeoff.
    
    %Trabalhos como o de \cite{Trindade2016} utilizam algoritmos genéticos para solucionar o problema de particionamento em SE.
    %Propõem novas abordagens usando técnicas de verificação baseadas nas teorias de módulo de satisfação (SMT, do inglês \textit{satisfiability modulo theories}).
    Works as the \cite{Trindade2016} use genetic algorithm to solve the partitioning problem of ES.
    Propose new approaches using techniques of verifications based on satisfiability modulo theories (SMT). 
    
    
    %concluindo tudo que foi dito
    %Os trabalhos citados buscam o estudo do desempenho e \design\ de SE no geral por meio de particionamento, mas nenhum com foco em \wearables.
    The cited works target the performance and design study of ES in general by partitioning, but none is aiming the wearables.
    
    
    %Em \cite{Jozwiak2017} os autores apresentam uma extensa revisão da literatura considerando vários aspectos de um SE, bem como suas tecnologias de \design,\ com foco em sistemas móveis modernos incluindo \wearables.
    In \cite{Jozwiak2017} the authors present an extensive revision of literature considering various aspects of an ES, as well as its technologies of design, aiming in mobile moderns systems including wearables.
    % wearable fpga
    %Além deste, é possível ver em \cite{Plessl2003, Ahola2007, Abdelhedi2016, Narumi2016, Lee2015} trabalhos que estudam \wearables\ junto de FPGA, mas nenhum deles utilizam a técnica de particionamento como meio para seu \design.
    Beside this, it is possible to see in \cite{Plessl2003, Ahola2007, Abdelhedi2016, Narumi2016, Lee2015} works that aim wearable with FPGA, but none uses the partitioning technique for its design.
    
    
    % conclusao
    %Este trabalho portanto consiste na análise do problema de particionamento \hs,\ com foco em \design\ de sistemas \wearables\ em plataforma FPGA.
    This work, therefore, consists in the analysis of hardware and software partitioning problem, with the focus in wearable systems design in FPGA platforms.
